
%%%%%%%%%%%%%%%%%%%%%%%%%%%%%%%%
% RECOMMENDER SYSTEMS
%%%%%%%%%%%%%%%%%%%%%%%%%%%%%%%%
\chapter{Recommender systems} \label{chap:recommender_systems}
In this chapter, we will briefly introduce in general what recommender systems are (hereinafter referred to as RS) and then continue with a description of the group variant of recommender systems and introduce common approaches and methods they employ.


\section{Recommender systems}\label{sec:01_rec_sys}
%\subsection{Introduction}
Broadly speaking, recommender systems are algorithms that are trying to suggest items to their users, or from another perspective, they aim to predict how would a user rate (like) an unseen item. They are used in a variety of settings, from e-commerce, media consumption, social networks, expert systems, search engines, and many others.

At their core as stated in \cite{wiki:Recommender_system} they are essentially an information filtering systems that aim to deal with selecting a subset from all the items by some filtration criteria, in this case the criteria is the user preference. They become necessary when it comes to suppressing the explosive growth in information on the web and function as a defence system against over loading the user with the vast amount of data that is present in almost every system today.

They can be views as decision support systems that guide users in finding and identifying items based on their idea about the desired state, in this situation the desired state is to find an item that they would like \cite{grouprecommendersystems_felfernig2018group}.

RS can provide both, by filtering based on user preference and providing alternatives by utilizing similarity. In a way, finding a suitable item can be viewed as a collaboration of the user and the recommender system, in varying degrees of freedom. From passively accepting the RS recommendations to actively interacting by giving feedback and stating the preferences.

\subsection{High-level examples}\label{subsec:01_rec_sys.high_level_examples}
Recommender systems are used in multiple ways, we now present a few high-level examples of where and how they can be used.
\begin{itemize}
    \item \textbf{Personalized merchandising}, where the system offers items that other users bought together with the viewed item, items that user could like based on previous orders or viewed items, either related or alternative choices.
    \item \textbf{Personalized content}, for content consumption services such as video and audio libraries. User is offered personalized content based on their preference profile, such as movies or videos that are similar to others they viewed, globally popular for a regional subset of the user-base and so on.
    \item \textbf{Personalized news feed and social media feed.} We want to offer user interesting content that would keep them engaged with the service. In the recent years there is push towards more social responsible RS design in this context due to the overwhelming power the social media have. It is important to deal with problems such as polarization \cite{recommender_systems_fighting_polarization}, fairness and disagreement.
    \item \textbf{Expert systems} that help doctors, operators and other people to make an informed decisions based on data. They can help to deal with data overload and filter relevant items and choices. As well as explore the item space when searching for solution with only weakly defined requirements.
    \item \textbf{Search experience}, that takes into consideration previous searches, preference profile, location and other attributes.
\end{itemize}


\subsection{Main algorithmic approaches}\label{subsec:01_rec_sys.main_alg_approaches}

We can generally divide them by their approach mentioned in \cite{RS_handbook-ricci2011} and \cite{constraint_based_recommenders} into:
\begin{itemize}
    \item \textbf{Collaborative filtering} (CF)\newline
        Solely based on ratings of items from users (user-item interactions). Trying to recommend unseen items that were liked by users who have a similar taste for other items that they both rated. And thus exploiting data of users with similar preferences.
    \item  \textbf{Content-based filtering} (CB)\newline
        Uses item features or item descriptions to recommend items similar to those that the user liked or interacted with. We are essentially building a model of preference for users and exploiting domain knowledge about items that match the users' model.
    \item \textbf{Constraint-based recommendation}\newline
        Depends on hand-crafted deep knowledge about items. User specifies a set of criteria based on which the system filters out items that meet the stated requirements. Additionally the system can sort the items based on their properties, if the stated criteria come with perceived importance - utility.
    \item \textbf{Hybrid systems}\newline    
        Combines multiple RS, either the same type with different parameters and/or different types together in order to increase recommendation accuracy. Main types according to \cite{grouprecommendersystems_felfernig2018group} are:
        \begin{itemize}
            \item Weighted where predictions of individual recommenders are summed up.
            \item Mixed, where predictions of individual recommenders are combined into one list.
            \item Cascade, where predictions of one recommender is feed as an input to another one.
        \end{itemize}
\end{itemize}

The popularity of the first two approaches varies from domain to domain. Some domains naturally contain item-specific data which allows using the \textit{content-based filtering} for example product parameters in e-shops, but other domains do not. Then it is more beneficial to use \textit{collaborative filtering} techniques or a mix of the two.

There are benefits and drawbacks for both, CF is able to extract latent meaning from the data that would remain inaccessible to CB that relies on items' features. But at the same time, it can cause problems to rely only on user-item interactions because we need a lot of data in order to make a precise recommendation. There will be nothing to recommend if we cannot find similar enough other users that already rated some unseen items. This problem is called a \textit{cold-start problem}.

Further, the third technique, \textit{Constraint-based filtering} requires a deep knowledge which describes items on a higher level and is not very interesting due to the algorithmic simplicity, we will thus not discuss it further.

One additional approach we didn't include in the list is \textit{Critique-based recommendation}. It's popularity is quite low, but still worth mentioning. It acts as sort of a guide through the item space. Where in cycles we show the user items that are distinct in some property (we could say they lie in different areas of the item space) and user either accepts or rejects them. Based on this feedback we narrow the user's preferences and offer additional set of items that reflect that and try to further guide the user to a satisfactory result.
\newline

Some of the classical and more advanced methods include:
\begin{itemize}
    \item User-based and item-based nearest neighbor similarity \cite{hill1995recommending}\cite{shardanand1995social}\cite{balabanovic1997fab}
    \item Matrix Factorisation techniques\cite{koren2009matrix}
    \item Deep Collaborative filtering \cite{he2017neural}\cite{YOUTUBE_deeprec-covington2016}\cite{DeepLearningBasedRecommenderSystem_zhang2019deep}
    \item Deep Content extraction\cite{DeepLearningBasedRecommenderSystem_zhang2019deep}
\end{itemize}



%%%%%%%%%%%%%%%%%%%%%%%%%%%%%%%%
% GROUP RECOMMENDER SYSTEMS
%%%%%%%%%%%%%%%%%%%%%%%%%%%%%%%%

\section{Group recommender systems} \label{sec:01_group_rec_sys}
\subsection{Introduction}\label{subsec:01_group_rec_sys.introduction}
So far we have discussed only recommender systems, where the object of a recommendation is a single user. But what do we do, when we have a group of users that we want to recommend to? For example, a group of friends selecting a movie that they want to watch together or a group listening to music? %We use a group recommender system (group RS). Where the objective of the system is to

Group recommender systems (group RS) are an interesting subarea of recommender systems, where the object of a recommendation is not just a single user but multiple individuals forming a group. Results of a recommendation for the group do need to reflect and balance individuals' preferences among all members.


\subsection{Characteristics of group RS}
\todo[inline]{mention some basics characteristics, how many recommended, feedback gathering, timeframe}


\todo[inline] {intro into next subchapters}


\subsection{Challenges}\label{subsec:01_group_rec_sys.challenges}
\begin{itemize}
    \item \textbf{How to merge individual preferences} \newline
        The main problem when extending RS systems to support the group setting is how to combine preferences of individual users together. It is possible to not support groups at all and let users deal with the act of combining them via discussion. But then the problem collapses back to single user setting, where the user is the whole group. Therefore we need to decide how and when to merge them. Main two approaches are mentioned bellow in \nameref{subsec:01_group_rec_sys.common_aproaches}.
    \item \textbf{Divergent group preferences} \newline
        Even in single variant systems there exist users so called \textit{Grey-sheep} and \textit{Black-sheep}, these users are hard to recommend to, because their preferences do not align with many or any other users (respectively). This problem is especially hard to solve in Collaborative filtering, which directly relies on finding similarity between users. And the same problem arises in the group setting, where it becomes much harder to find solutions that would be satisfactory to all of its members. So in Group RS the problem of outlining users can be observed on two levels, in the usual situation, where the groups aggregated preferences are outlined and on another level where the inter-group preferences of individual users do not match.
    \item \textbf{Feedback gathering} \newline
        In most applications feedback is gathered directly as well as indirectly. Directly by users rating recommended items, and indirectly by observing users behavior such as which items they've visited or how long they've interacted with the item. Gathering direct feedback in the group setting is still possible, even if harder due to possibility that not all members leave a rating, but gathering the indirect, implicit feedback can become even impossible, depending on how the system-user interactions are designed. In most cases users will be selecting an item on a single device, under account of one person, therefore it is hard to distinguish what are preferences of that one individual and what are preferences of the group.
    \item \textbf{Active/passive, primary/secondary group members} \newline
        Another interesting issue arises when we consider that possibly not all members are equally important when it comes to the recommendation as mentioned in \cite{deCampos_2009_managing_uncertanity_in_grouprec}. One example could be when parents select a movie to watch with their children, the children should arguably be given a priority over the selection. Second example could be that we would possibly want to prioritize satisfaction of individual in the group that were less satisfied the last time an item was consumed.
        
    \item \textbf{How to explain provided recommendations} \newline
    
    \item \textbf{Not all members present} \newline
    \item \textbf{Preference gathering} \newline
    neni to moc podobne tomu feedback gathering?
    \item \textbf{Selecting from the provided list} \newline
\end{itemize}


\subsection{Classification}
As found in \cite{masthoff_2011_group_rec_systems}
\begin{itemize}
    \item pref known or gathered overtime
    \item just presented or experienced in real time
    \item group is passive or active 
    \item recommending single item, set, list, sequence?
    \todo[inline]{We need to distinguish what the result will be used for, generating the recommendations for a list where a single item will be selected is different from when we generate a list that will be consumed in item after item fashion. Is there a naming for this? Single item vs K-item utility?? \cite{xiao_2017_fairness_aware_g_rec} and the original paper \cite{connor_2001_polylens_rec_for_groups}}
\end{itemize}


\subsection{Common approaches}\label{subsec:01_group_rec_sys.common_aproaches}
\todo[inline]{Mention ephemeral vs persistent from paper PolyLens: A Recommender System for
Groups of Users}
Now we will introduce the two main algorithmic approaches of group recommender systems, according to \cite{recommendations_to_groups-jameson2007} these are:
\begin{itemize}
    \item \textbf{Group aware RS approach}\newline
    Builds a model for the group based on the preferences of all of its members. Either directly by creating a model of preference for the group or by aggregating models of individual users together and then recommending items for the group as a single entity.
    \item \textbf{RS aggregation approach}\newline
    Use single-user RS to recommend to each individual of the group and then aggregate the results together to create the final recommendation for the group.
    
\end{itemize}

\todo[inline]{Mention division into groups based on where the recommender is "located" in relation to the aggregation, in short and then ref. the reader to the related work chapter where we discuss in depth. Or move the division from the related work here?}

In the RS aggregation approach we further distinguish between situations where we have predictions for all possible items and therefore can do aggregation directly on the ratings of all items or if only have a list of recommendation for each user - subset of all the items. These two can function very differently, for example taking in context only the position in the recommended list or position and the rating. They are mentioned separately in \cite{recommendations_to_groups-jameson2007}, but the approaches are very similar, they only differ in the availability of provided results from the underlying RS, so we group them together under one main direction.

Further, both group aware RS and aggregation approaches do both have some advantages and disadvantages. One of advantages of the Aggregation approach is that we can use the same RS as we would use for an individual recommendation, either as a black box aggregating directly the top items provided, or in more involved way by utilizing the predicted ratings. On the other hand, the aggregation strategies do rely on single-user RS so there is not much that can be done in order to extract some hidden latent preferences of the group, which in case of the first method, the group aware approach, can potentially be extracted.


We will go in-depth to discuss techniques used in the latest literature in chapter \ref{chap:related_work}.


At the same time, we need to define what does it even mean to recommend something to a group. Do we measure it by fairness, overall user satisfaction, or by the least satisfied member of the group? We will go in depth to describe common approaches to these problems in chapter \ref{chap:fairness}.

