\chapter*{Introduction}

The problem of fairness can be can be viewed from many different angles. And it can shape how we design our algorithms and how we evaluate their value. One of those systems are algorithms creating recommendations, specifically to this thesis, the group recommendation variant.

\section*{Problem statement}
The current research on the topic of group recommender systems is lacking. There are no standardized data sets that would offer evaluation of the research without using various methods of data augmentation and artificial data creation.
And the definition of fairness is not unified. It can mean many different things and be evaluated in many different methods. 


With these two aforementioned problems goes hand in hand the very subjective nature of user preference.

\section*{Research objective}

We would like to research how fairness can be measured, defined and eventually used to improve recommendations in the group setting. And if we could adapt fairness preserving methods from other fields to group recommendation problem.

\section*{Thesis structure}

We start with an introduction to recommender systems and specifically to group recommender systems in chapter \ref{chap01}. Then we will continue with the definitions and evaluation methods for fairness in chapter \ref{chap02}. Next we will introduce few algorithms that are used in the group recommender field in chapter \ref{chap03}.
\textcolor{red}{TODO: check out other works and decide what should be here. This can be nice from the reading perspective, but is it really necessary?}

\addcontentsline{toc}{chapter}{Introduction}